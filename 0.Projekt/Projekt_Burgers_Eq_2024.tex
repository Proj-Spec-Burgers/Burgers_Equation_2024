\documentclass[a4paper,12pt]{article}

% Ustawienia polskich znaków i kodowania
\usepackage[utf8]{inputenc}
\usepackage[T1]{fontenc}
\usepackage[polish]{babel}

% Ustawienia marginesów
\usepackage{geometry}
\geometry{left=2.5cm,right=2.5cm,top=2.5cm,bottom=2.5cm}

% Pakiety dodatkowe
\usepackage{amsmath}
\usepackage{graphicx}
\usepackage{hyperref}

% Tytuł, autor, data
\title{Numeryczne wyznaczanie rozwiązań równania Burgersa przy pomocy metody różnic skończonych}
\author{Konrad Bonicki, Tomasz Orzechowski, Maciej Pestka}
\date{\today}

\begin{document}

\maketitle



\section{Wprowadzenie}
W tej pracy przybliżę równanie Burgersa oraz jego zastosowania, jego rozwiązanie analityczne i metody, które użyliśmy podczas pisania programu, który numerycznie wyznacza rozwiązania przytoczonego równania.

\section{Równanie Burgersa}
Równanie Burgersa w ogólnej postaci jest fundamentalnym, nieliniowym, parabolicznym równaniem różniczkowym cząstkowym drugiego rzędu występującym w przeróżnych obszarach zastosowań matematyki, takich jak mechanika płynów, dynamika gazów oraz płynność ruchu:
     \begin{equation}
\frac{\partial u}{\partial t} + u \frac{\partial u}{\partial x} = v \frac{\partial ^2 u}{\partial x^2}
\end{equation}
gdzie: \\ t - zmienna niezależna zwykle interpretowana jako czas, \\ x - zmienna niezależna zwykle interpretowana jako położenie, \\ u(x,t) - zmienna zależna zwykle interpretowana jako prędkość płynu, \\ v - stały parametr, zwykle interpretowany jako lepkość płynu

\section{Transformacja Hopf-Cole'a}
$$u(x,t)=-2v\frac{\theta _x (x,t)}{\theta (x,t)}$$
$$u=-2v\frac{\theta _x}{\theta}$$
	$$u_x=-2v\frac{\theta _{xx} \theta - \theta ^{2}_{x}}{\theta ^2}, \quad u_t=-2v\frac{\theta _{xt}\theta - \theta _x \theta _t}{\theta ^2}$$
	$$u_{xx}=-2v\frac{\theta _{xxx}\theta ^2 - 3\theta \theta _x \theta _{xx} + 2\theta ^{3}_{x}}{\theta ^3}$$
	$$u_t + u u_x = v u_{xx}$$
$$-2v \frac{\theta _{xt}\theta - \theta _x \theta _t }{\theta ^2} + 4 v^2 \frac{\theta \theta _x \theta _{xx} - \theta ^{3}_{x}}{\theta ^3} = -2v^2\frac{\theta _{xxx} \theta ^2 - 3\theta \theta _x \theta _{xx} + 2\theta ^{3}_{x}}{\theta ^3}$$
$$-2v\theta (\theta _{xt} \theta - \theta _x \theta _t ) +4v^2 (\theta \theta _x \theta _{xx} - \theta ^{3}_{x}) = -2v^2(\theta _{xxx} \theta ^2 - 3\theta \theta _x \theta _{xx} + 2\theta ^{3}_{x})$$
$$\theta ^2 \theta _{xt} - \theta \theta _x \theta _t -2v \theta \theta _x \theta _{xx} +2\theta^{3}_{x}=v \theta _{xxx} \theta ^2 -3v\theta \theta _x \theta _{xx} +2 \theta ^{3}_{x}$$
$$\theta \theta _{xt} - \theta _x \theta _t -2v\theta _x \theta _{xx}= v\theta \theta _{xxx} - 3v\theta _x \theta _{xx}$$
$$\theta \theta _{xt} - \theta _x \theta _t = v(\theta \theta _{xxx} - \theta _x \theta _{xx})$$
$$\theta \frac{\partial 
\theta ^2}{\partial x \partial t}- \frac{\partial \theta}{\partial x} \frac{\partial \theta}{\partial t} = v(\theta \frac{\partial ^3 \theta}{\partial x^3}-\frac{\partial \theta}{\partial x}\frac{\partial ^2 \theta}{\partial x^2})$$
$$\frac{\partial \theta}{\partial t}(\theta \frac{\partial \theta}{\partial x}-\frac{\partial \theta}{\partial x})=v\frac{\partial ^2 \theta}{\partial x^2}(\theta \frac{\partial \theta}{\partial x}-\frac{\partial \theta}{\partial x})$$
$$\theta \theta _x - \theta _x \neq 0$$
$$ \theta _t = v \theta _{xx}$$
Po podstawieniu transformacji Hopf-Cole'a do równania Burgersa otrzymaliśmy równanie ciepła.

\section{Rozwiązanie analityczne przy pomocy transformacji Hopf-Cole'a}
Równanie Burgersa: $$u_t +uu_x=vu_{xx}$$
Warunek początkowy: $$u(x,0)=\sin (\pi x), \quad 0<x<1,$$ Warunki brzegowe: $$u(0,t)=u(1,t)=0, \quad t>0. $$ Transformacja Hopf-Cole'a: $$u(x,t)=-2v\frac{\theta _x (x,t)}{\theta (x,t)}$$ Z równania Burgersa z poprzedniego punktu mamy: $$ \theta _t = v \theta _{xx}, \quad 0<x<1 \quad t>0$$
Warunek początkowy $\theta :$
$$u(x,0)=-2v\frac{\theta _x (x,0)}{\theta (x,0)}, \quad \theta (x,0) \neq 0$$
$$\sin (\pi x) = -2v\frac{\theta _x (x,0)}{\theta (x,0)}$$
$$-\frac{1}{2v}\sin (\pi x)=\frac{\theta _x (x,0)}{\theta (x,0)}$$
$$-\frac{1}{2v} \int^x_0 \sin (\pi x)dx = \int^x_0 \frac{\theta _x (x,0)}{\theta (x,0)}dx $$
$$-\frac{1}{2v} [-\frac{1}{\pi}\cos (\pi x)]^{x=x}_0=[ln[\theta (x,0)]]^{x=x}_0$$
$$\frac{1}{2v\pi}(\cos (\pi x)-1)=ln[\theta (x,0)]-ln[\theta (0,0)]$$
Wyznaczamy $\theta (0,0)$:
$$u(0,0)=-2v\frac{\theta _x (0,0)}{\theta (0,0)}$$
$$u(0,0)=\sin (\pi 0)=0$$
$$0=\frac{\theta _x (0,0)}{\theta (0,0)}$$
$$\int^x_0 0 dx = \int^x_0 \frac{\theta _x (0,0)}{\theta (0,0)} dx$$
$$ln(\theta(0,0))=0$$
$$\theta (0,0) = e^0=1$$
Co daje:
$$\theta (x,0)= e^{\frac{1}{2v\pi}(\cos (\pi x)-1)}, \quad 0<x<1$$
Warunki brzegowe $\theta:$
$$u(0,t)=u(1,t)=0$$
$$-2v\frac{\theta _x (0,t)}{\theta (0,t)}=-2v\frac{\theta _x (1,t)}{\theta (1,t)}=0,\quad \theta (0,t)\neq 0 \quad \theta (1,t) \neq 0$$
$$\theta _x (0,t) = \theta _x (1,t) =0, \quad t>0$$
Rozwiązanie: $$\theta _t = v \theta _{xx}, \quad 0<x<1 \quad t>0$$
Do rozwiązania tego równania wykorzystam metodę Fouriera:
$$\theta (x,t)=X(x)T(t)$$
$$X(x)T`(t)=vX``(x)T(t)$$
$$\frac{T`(t)}{vT(t)}=\frac{X``(x)}{X(x)}=-\lambda ^2$$
$$T`(t)=-\lambda ^2 vT(t)$$
$$T(t)=e^{-\lambda ^2 vt}$$
$$X``(x)=-\lambda ^2X(x)$$
$$X(x)=A\sin(\lambda x)+B\cos(\lambda x)$$
$$\theta (x,t) = (A\sin(\lambda x)+B\cos(\lambda x))e^{-\lambda ^2 vt}$$
$$\theta _x (x,t)=(A\cos(\lambda x)-B\sin(\lambda x))e^{-\lambda ^2 vt}$$
$$\theta _x (0,t)=0$$
$$(A\cos(\lambda 0)-B\sin(\lambda 0))e^{-\lambda ^2 vt}=0$$
$$Ae^{-\lambda ^2 vt}=0 <=> A =0$$
$$\theta _x (1,t)=0$$
$$(-B\sin(\lambda))e^{-\lambda ^2 vt}=0$$
Funkcja sinus przyjmuje wartości 0 dla $\lambda=n\pi,\quad n=0,1,2,...$, omijamy rozwiązanie trywialne $B=0$:
$$\frac{\partial \theta _n}{\partial x} (x,t) = \sum_{n=0}^{\infty}-B_n \sin (n\pi x) e^{-n^2 \pi ^2 vt} $$
Wracamy z wyznaczonymi współczynnikami do momentu przed różniczkowaniem funkcji $\theta$:
$$\theta _n (x,t) =B_0+ \sum_{n=1}^\infty B_n \cos (n\pi x) e^{-n^2 \pi ^2 vt}$$
Wyznaczamy współczynniki $B_n\quad (n=0,1,2,...)$:
$$B_0 = \int_0^1 e^{\frac{1}{2v\pi}(\cos (\pi x)-1)}dx 
=e^{-\frac{1}{2v\pi}}\int_0^1 e^{\frac{\cos (\pi x)}{2v\pi}}dx$$



\end{document}
