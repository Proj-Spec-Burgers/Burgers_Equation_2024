\documentclass[a4paper,12pt]{article}
\usepackage[utf8]{inputenc}
\usepackage[T1]{fontenc}
\usepackage{times}
\usepackage{hyperref}

\title{Raport \\ Numeryczne wyznaczanie rozwiązań równania Burgersa przy pomocy różnic skończonych}
\author{Konrad Bonicki, Tomasz Orzechowski, Maciej Pestka \\ Promotor: dr inż. Paweł Wojda }
\date{\today}

\begin{document}

\maketitle

\section*{Cel projektu}
Celem pracy jest zapoznanie z równaniem Burgersa oraz jego zastosowaniami oraz wyznaczenie jego rozwiązań numerycznych. Zadanie polega na zaprojektowaniu, napisaniu i przetestowaniu programu służącego do obliczania rozwiązań równania Burgersa. 

\section*{Opis projektu}
Przedstawione i omówione zostały:
\begin{itemize}
\item Równanie Burgersa
\item Transformacja Hopf-Cole’a
\item Rozwiązanie analityczne przy pomocy transformacji Hopf-Cole’a
\item Metoda różnic skończonych
\item Metoda Eulera 
\item Metoda Rungego-Kutty
\item Rozwiązanie numeryczne równania Burgersa
\item Rozwiązanie numeryczne transformaty Hopf-Cole’a
\end{itemize}

\section*{Wnioski}
Rozwiązane zostały numerycznie RRC równania transportu, nielepkiego równania Burgersa, równania ciepła, równania Burgersa. Użyto metod Eulera, Rungego-Kutty drugiego rzędu dla wszystkich równań oraz metody niejawnej dla równania ciepła oraz równania Burgersa dla warunku początkowego $u_{x_{i},0}=e^{-(x-5)^{2}}$. Przedstawiono wyniki za pomocą wykresów dla różnych wartości czasów $t_{j}$ aby zobrazować jak działa liniowość/nieliniowość równań oraz jaki wpływ ma współczynnik $\beta$ lepkości na rozwiązanie. Pokazano efekt „fali uderzeniowej” nieliniowości oraz znalezionio przedział rzeczywisty $\beta$ w którym rozwiązanie przestaje być stabilne. Program wykonujący obliczenia napisano w języku C++.\\
Wykonano analizę transformaty Hopf-Cole’a dla warunku początkowego $u(x_{i},0)=sin(\pi x_{i})$. Pokazano jak zachowuje się rozwiązanie dla różnego wyboru kroku przestrzennego $x_{i}$, kroku czasowego $t_{j}$ oraz różnych wartości $\beta$. Program wykonujący obliczenia napisano w języku C++.

\section*{Podział pracy}
Konrad Bonicki: wstęp teoretyczny. \\ 
Tomasz Orzechowski: RRC transportu, nielepkiego Równania Burgersa, ciepła oraz równanie Burgersa - program w C++ oraz analiza wyników.\\
Maciej Pestka: pisane kodu do transformacji Hopf-Cole'a.

\begin{thebibliography}{9}
\bibitem{1} W. Rudin, \textit{„Podstawy analizy matematycznej”, PWN, Warszawa 2009.}
\bibitem{2} David Kincaid, Ward Cheney,    \textit{ Analiza numeryczna, 2006}
\bibitem{3} S.Kutluay, A.R. Bahadir, A.Özdes, \textit{Numerical solution of one-dimensional Burgers equation: explicit and exact-explicit finite difference methods, 1999}
\bibitem{3} A. Salih, \textit{ Burgers’ Equation, 2016}
\end{thebibliography}

\end{document}