\documentclass[a4paper,12pt]{article}
\usepackage[utf8]{inputenc}
\usepackage[T1]{fontenc}
\usepackage[polish]{babel}
\usepackage{times}
\usepackage{hyperref}

\title{Raport \\ Numeryczne wyznaczanie rozwiązań równania Burgersa przy pomocy różnic skończonych}
\author{Konrad Bonicki, Tomasz Orzechowski, Maciej Pestka \\ Promotor: dr inż. Paweł Wojda }
\date{\today}

\begin{document}

\maketitle

\section*{Cel projektu}
Celem pracy jest zapoznanie z równaniem Burgersa oraz jego zastosowaniami oraz wyznaczenie jego rozwiązań numerycznych. Zadanie polega na zaprojektowaniu, napisaniu i przetestowaniu programu służącego do obliczania rozwiązań równania Burgersa. 

\section*{Opis projektu}


\section*{Wnioski}


\section*{Podział pracy}
Konrad Bonicki - wstęp teoretyczny \\ 
Tomasz Orzechowski - \\
Maciej Pestka - Pisane kodu do transformacja Hopf-Cole'a 

\begin{thebibliography}{9}
\bibitem{1} S.Kutluay, A.R. Bahadir, A.Özdes, \textit{Numerical solution of one-dimensional Burgers equation: explicit and exact-explicit finite difference methods}
\bibitem{2} David Kincaid, Ward Cheney,    \textit{ Analiza numeryczna}
\bibitem{3}en.wikipedia.org/wiki/
\end{thebibliography}

\end{document}
